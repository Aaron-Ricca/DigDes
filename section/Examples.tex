\section{Examples}


% ==================== FILP FLOP ========================
    \subsection{Flip flop}
    \begin{lstlisting}[language=VHDL, escapeinside={(*@}{@*)}]
entity d_ff_rst is
    port(
        clk : in bit;
        rst : in bit;
        d : in bit;
        q : out bit
        );
end d_ff_rst;
    
architecture behavioral of d_ff_rst is
begin
    register : process(clk, rst)
            -- d fehlt in Sens.list
    begin   -- Nur clk und Reset
            -- aktivieren Prozess
        if (rst = '1') then
            -- Asynchroner Reset
            q <= '0';
            -- wird zuerst abgearbeitet
        elsif (clk(*@'@*)event and clk = '1') then (*@\setlength{\fboxsep}{1pt}\colorbox{yellow}{\strut\textbf{\textcolor[rgb]{0.0,0.4,0.2}{-- Sintassi per la detezione del fianco salita del clock}}}@*)
            q <= d; -- Synchroner Teil
        end if; -- Kein abschliessendes
        -- else: in allen anderen
    end process; -- Fällen wird gespeichert.
end behavioral;
    \end{lstlisting}


% ==================== TEST BENCH ========================
    \subsection{Test bench (dt)}
        \begin{lstlisting}[language=VHDL, escapeinside={(*@}{@*)}]
-- Library declarations
library ieee;
use ieee.std_logic_1164.all;
use ieee.numeric_std.all;

-- entity declarations
entity exor_tb is
end exor_tb;

-- architecture declarations
architecture tb of exor_tb is
    -- constant declarations
    constant sim_cyc : time := 10 ns;
    -- signal declarations
    signal tb_a, tb_b : std_ulogic;
    signal tb_y: std_ulogic;
    -- Component declarations
    component xor2
        port(
        xor2_a, xor2_b : in std_ulogic;
        xor2_y
        : out std_ulogic);
    end component;
    -- configuration
    for all : xor2 use entity work.xor2(behavioral);

    begin
        -- instance assignments
        dut : xor2
            port map(
            xor2_a => tb_a,
            xor2_b => tb_b,
            xor2_y => tb_y
            );

        -- Signal assignments for stimuli
        Stimuli : process
            variable tb_in_vec: std_ulogic_vector(2 - 1 downto 0);
        begin
            for i in 0 to (2 ** 2) - 1 loop
            tb_in_vec := std_ulogic_vector(to_unsigned(i, 2));
            tb_a <= tb_in_vec(1);
            tb_b <= tb_in_vec(0);
            wait for sim_cyc;
            end loop;
            wait;
        end process;

        -- Evaluation of responses
        Response : process
        begin
            wait for (sim_cyc - 1 ns);
            assert (tb_y = '0') report "error at vector 00" severity error;
            wait for sim_cyc;
            assert (tb_y = '1') report "error at vector 01" severity error;
            wait for sim_cyc;
            assert (tb_y = '1') report "error at vector 10" severity error;
            wait for sim_cyc;
            assert (tb_y = '0') report "error at vector 11" severity error;
        end process;
end tb;
            \end{lstlisting}